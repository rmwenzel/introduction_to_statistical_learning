
% Default to the notebook output style

    


% Inherit from the specified cell style.




    
\documentclass[11pt]{article}

    
    
    \usepackage[T1]{fontenc}
    % Nicer default font (+ math font) than Computer Modern for most use cases
    \usepackage{mathpazo}

    % Basic figure setup, for now with no caption control since it's done
    % automatically by Pandoc (which extracts ![](path) syntax from Markdown).
    \usepackage{graphicx}
    % We will generate all images so they have a width \maxwidth. This means
    % that they will get their normal width if they fit onto the page, but
    % are scaled down if they would overflow the margins.
    \makeatletter
    \def\maxwidth{\ifdim\Gin@nat@width>\linewidth\linewidth
    \else\Gin@nat@width\fi}
    \makeatother
    \let\Oldincludegraphics\includegraphics
    % Set max figure width to be 80% of text width, for now hardcoded.
    \renewcommand{\includegraphics}[1]{\Oldincludegraphics[width=.8\maxwidth]{#1}}
    % Ensure that by default, figures have no caption (until we provide a
    % proper Figure object with a Caption API and a way to capture that
    % in the conversion process - todo).
    \usepackage{caption}
    \DeclareCaptionLabelFormat{nolabel}{}
    \captionsetup{labelformat=nolabel}

    \usepackage{adjustbox} % Used to constrain images to a maximum size 
    \usepackage{xcolor} % Allow colors to be defined
    \usepackage{enumerate} % Needed for markdown enumerations to work
    \usepackage{geometry} % Used to adjust the document margins
    \usepackage{amsmath} % Equations
    \usepackage{amssymb} % Equations
    \usepackage{textcomp} % defines textquotesingle
    % Hack from http://tex.stackexchange.com/a/47451/13684:
    \AtBeginDocument{%
        \def\PYZsq{\textquotesingle}% Upright quotes in Pygmentized code
    }
    \usepackage{upquote} % Upright quotes for verbatim code
    \usepackage{eurosym} % defines \euro
    \usepackage[mathletters]{ucs} % Extended unicode (utf-8) support
    \usepackage[utf8x]{inputenc} % Allow utf-8 characters in the tex document
    \usepackage{fancyvrb} % verbatim replacement that allows latex
    \usepackage{grffile} % extends the file name processing of package graphics 
                         % to support a larger range 
    % The hyperref package gives us a pdf with properly built
    % internal navigation ('pdf bookmarks' for the table of contents,
    % internal cross-reference links, web links for URLs, etc.)
    \usepackage{hyperref}
    \usepackage{longtable} % longtable support required by pandoc >1.10
    \usepackage{booktabs}  % table support for pandoc > 1.12.2
    \usepackage[inline]{enumitem} % IRkernel/repr support (it uses the enumerate* environment)
    \usepackage[normalem]{ulem} % ulem is needed to support strikethroughs (\sout)
                                % normalem makes italics be italics, not underlines
    

    
    
    % Colors for the hyperref package
    \definecolor{urlcolor}{rgb}{0,.145,.698}
    \definecolor{linkcolor}{rgb}{.71,0.21,0.01}
    \definecolor{citecolor}{rgb}{.12,.54,.11}

    % ANSI colors
    \definecolor{ansi-black}{HTML}{3E424D}
    \definecolor{ansi-black-intense}{HTML}{282C36}
    \definecolor{ansi-red}{HTML}{E75C58}
    \definecolor{ansi-red-intense}{HTML}{B22B31}
    \definecolor{ansi-green}{HTML}{00A250}
    \definecolor{ansi-green-intense}{HTML}{007427}
    \definecolor{ansi-yellow}{HTML}{DDB62B}
    \definecolor{ansi-yellow-intense}{HTML}{B27D12}
    \definecolor{ansi-blue}{HTML}{208FFB}
    \definecolor{ansi-blue-intense}{HTML}{0065CA}
    \definecolor{ansi-magenta}{HTML}{D160C4}
    \definecolor{ansi-magenta-intense}{HTML}{A03196}
    \definecolor{ansi-cyan}{HTML}{60C6C8}
    \definecolor{ansi-cyan-intense}{HTML}{258F8F}
    \definecolor{ansi-white}{HTML}{C5C1B4}
    \definecolor{ansi-white-intense}{HTML}{A1A6B2}

    % commands and environments needed by pandoc snippets
    % extracted from the output of `pandoc -s`
    \providecommand{\tightlist}{%
      \setlength{\itemsep}{0pt}\setlength{\parskip}{0pt}}
    \DefineVerbatimEnvironment{Highlighting}{Verbatim}{commandchars=\\\{\}}
    % Add ',fontsize=\small' for more characters per line
    \newenvironment{Shaded}{}{}
    \newcommand{\KeywordTok}[1]{\textcolor[rgb]{0.00,0.44,0.13}{\textbf{{#1}}}}
    \newcommand{\DataTypeTok}[1]{\textcolor[rgb]{0.56,0.13,0.00}{{#1}}}
    \newcommand{\DecValTok}[1]{\textcolor[rgb]{0.25,0.63,0.44}{{#1}}}
    \newcommand{\BaseNTok}[1]{\textcolor[rgb]{0.25,0.63,0.44}{{#1}}}
    \newcommand{\FloatTok}[1]{\textcolor[rgb]{0.25,0.63,0.44}{{#1}}}
    \newcommand{\CharTok}[1]{\textcolor[rgb]{0.25,0.44,0.63}{{#1}}}
    \newcommand{\StringTok}[1]{\textcolor[rgb]{0.25,0.44,0.63}{{#1}}}
    \newcommand{\CommentTok}[1]{\textcolor[rgb]{0.38,0.63,0.69}{\textit{{#1}}}}
    \newcommand{\OtherTok}[1]{\textcolor[rgb]{0.00,0.44,0.13}{{#1}}}
    \newcommand{\AlertTok}[1]{\textcolor[rgb]{1.00,0.00,0.00}{\textbf{{#1}}}}
    \newcommand{\FunctionTok}[1]{\textcolor[rgb]{0.02,0.16,0.49}{{#1}}}
    \newcommand{\RegionMarkerTok}[1]{{#1}}
    \newcommand{\ErrorTok}[1]{\textcolor[rgb]{1.00,0.00,0.00}{\textbf{{#1}}}}
    \newcommand{\NormalTok}[1]{{#1}}
    
    % Additional commands for more recent versions of Pandoc
    \newcommand{\ConstantTok}[1]{\textcolor[rgb]{0.53,0.00,0.00}{{#1}}}
    \newcommand{\SpecialCharTok}[1]{\textcolor[rgb]{0.25,0.44,0.63}{{#1}}}
    \newcommand{\VerbatimStringTok}[1]{\textcolor[rgb]{0.25,0.44,0.63}{{#1}}}
    \newcommand{\SpecialStringTok}[1]{\textcolor[rgb]{0.73,0.40,0.53}{{#1}}}
    \newcommand{\ImportTok}[1]{{#1}}
    \newcommand{\DocumentationTok}[1]{\textcolor[rgb]{0.73,0.13,0.13}{\textit{{#1}}}}
    \newcommand{\AnnotationTok}[1]{\textcolor[rgb]{0.38,0.63,0.69}{\textbf{\textit{{#1}}}}}
    \newcommand{\CommentVarTok}[1]{\textcolor[rgb]{0.38,0.63,0.69}{\textbf{\textit{{#1}}}}}
    \newcommand{\VariableTok}[1]{\textcolor[rgb]{0.10,0.09,0.49}{{#1}}}
    \newcommand{\ControlFlowTok}[1]{\textcolor[rgb]{0.00,0.44,0.13}{\textbf{{#1}}}}
    \newcommand{\OperatorTok}[1]{\textcolor[rgb]{0.40,0.40,0.40}{{#1}}}
    \newcommand{\BuiltInTok}[1]{{#1}}
    \newcommand{\ExtensionTok}[1]{{#1}}
    \newcommand{\PreprocessorTok}[1]{\textcolor[rgb]{0.74,0.48,0.00}{{#1}}}
    \newcommand{\AttributeTok}[1]{\textcolor[rgb]{0.49,0.56,0.16}{{#1}}}
    \newcommand{\InformationTok}[1]{\textcolor[rgb]{0.38,0.63,0.69}{\textbf{\textit{{#1}}}}}
    \newcommand{\WarningTok}[1]{\textcolor[rgb]{0.38,0.63,0.69}{\textbf{\textit{{#1}}}}}
    
    
    % Define a nice break command that doesn't care if a line doesn't already
    % exist.
    \def\br{\hspace*{\fill} \\* }
    % Math Jax compatability definitions
    \def\gt{>}
    \def\lt{<}
    % Document parameters
    \title{ch07\_notes}
    
    
    

    % Pygments definitions
    
\makeatletter
\def\PY@reset{\let\PY@it=\relax \let\PY@bf=\relax%
    \let\PY@ul=\relax \let\PY@tc=\relax%
    \let\PY@bc=\relax \let\PY@ff=\relax}
\def\PY@tok#1{\csname PY@tok@#1\endcsname}
\def\PY@toks#1+{\ifx\relax#1\empty\else%
    \PY@tok{#1}\expandafter\PY@toks\fi}
\def\PY@do#1{\PY@bc{\PY@tc{\PY@ul{%
    \PY@it{\PY@bf{\PY@ff{#1}}}}}}}
\def\PY#1#2{\PY@reset\PY@toks#1+\relax+\PY@do{#2}}

\expandafter\def\csname PY@tok@w\endcsname{\def\PY@tc##1{\textcolor[rgb]{0.73,0.73,0.73}{##1}}}
\expandafter\def\csname PY@tok@c\endcsname{\let\PY@it=\textit\def\PY@tc##1{\textcolor[rgb]{0.25,0.50,0.50}{##1}}}
\expandafter\def\csname PY@tok@cp\endcsname{\def\PY@tc##1{\textcolor[rgb]{0.74,0.48,0.00}{##1}}}
\expandafter\def\csname PY@tok@k\endcsname{\let\PY@bf=\textbf\def\PY@tc##1{\textcolor[rgb]{0.00,0.50,0.00}{##1}}}
\expandafter\def\csname PY@tok@kp\endcsname{\def\PY@tc##1{\textcolor[rgb]{0.00,0.50,0.00}{##1}}}
\expandafter\def\csname PY@tok@kt\endcsname{\def\PY@tc##1{\textcolor[rgb]{0.69,0.00,0.25}{##1}}}
\expandafter\def\csname PY@tok@o\endcsname{\def\PY@tc##1{\textcolor[rgb]{0.40,0.40,0.40}{##1}}}
\expandafter\def\csname PY@tok@ow\endcsname{\let\PY@bf=\textbf\def\PY@tc##1{\textcolor[rgb]{0.67,0.13,1.00}{##1}}}
\expandafter\def\csname PY@tok@nb\endcsname{\def\PY@tc##1{\textcolor[rgb]{0.00,0.50,0.00}{##1}}}
\expandafter\def\csname PY@tok@nf\endcsname{\def\PY@tc##1{\textcolor[rgb]{0.00,0.00,1.00}{##1}}}
\expandafter\def\csname PY@tok@nc\endcsname{\let\PY@bf=\textbf\def\PY@tc##1{\textcolor[rgb]{0.00,0.00,1.00}{##1}}}
\expandafter\def\csname PY@tok@nn\endcsname{\let\PY@bf=\textbf\def\PY@tc##1{\textcolor[rgb]{0.00,0.00,1.00}{##1}}}
\expandafter\def\csname PY@tok@ne\endcsname{\let\PY@bf=\textbf\def\PY@tc##1{\textcolor[rgb]{0.82,0.25,0.23}{##1}}}
\expandafter\def\csname PY@tok@nv\endcsname{\def\PY@tc##1{\textcolor[rgb]{0.10,0.09,0.49}{##1}}}
\expandafter\def\csname PY@tok@no\endcsname{\def\PY@tc##1{\textcolor[rgb]{0.53,0.00,0.00}{##1}}}
\expandafter\def\csname PY@tok@nl\endcsname{\def\PY@tc##1{\textcolor[rgb]{0.63,0.63,0.00}{##1}}}
\expandafter\def\csname PY@tok@ni\endcsname{\let\PY@bf=\textbf\def\PY@tc##1{\textcolor[rgb]{0.60,0.60,0.60}{##1}}}
\expandafter\def\csname PY@tok@na\endcsname{\def\PY@tc##1{\textcolor[rgb]{0.49,0.56,0.16}{##1}}}
\expandafter\def\csname PY@tok@nt\endcsname{\let\PY@bf=\textbf\def\PY@tc##1{\textcolor[rgb]{0.00,0.50,0.00}{##1}}}
\expandafter\def\csname PY@tok@nd\endcsname{\def\PY@tc##1{\textcolor[rgb]{0.67,0.13,1.00}{##1}}}
\expandafter\def\csname PY@tok@s\endcsname{\def\PY@tc##1{\textcolor[rgb]{0.73,0.13,0.13}{##1}}}
\expandafter\def\csname PY@tok@sd\endcsname{\let\PY@it=\textit\def\PY@tc##1{\textcolor[rgb]{0.73,0.13,0.13}{##1}}}
\expandafter\def\csname PY@tok@si\endcsname{\let\PY@bf=\textbf\def\PY@tc##1{\textcolor[rgb]{0.73,0.40,0.53}{##1}}}
\expandafter\def\csname PY@tok@se\endcsname{\let\PY@bf=\textbf\def\PY@tc##1{\textcolor[rgb]{0.73,0.40,0.13}{##1}}}
\expandafter\def\csname PY@tok@sr\endcsname{\def\PY@tc##1{\textcolor[rgb]{0.73,0.40,0.53}{##1}}}
\expandafter\def\csname PY@tok@ss\endcsname{\def\PY@tc##1{\textcolor[rgb]{0.10,0.09,0.49}{##1}}}
\expandafter\def\csname PY@tok@sx\endcsname{\def\PY@tc##1{\textcolor[rgb]{0.00,0.50,0.00}{##1}}}
\expandafter\def\csname PY@tok@m\endcsname{\def\PY@tc##1{\textcolor[rgb]{0.40,0.40,0.40}{##1}}}
\expandafter\def\csname PY@tok@gh\endcsname{\let\PY@bf=\textbf\def\PY@tc##1{\textcolor[rgb]{0.00,0.00,0.50}{##1}}}
\expandafter\def\csname PY@tok@gu\endcsname{\let\PY@bf=\textbf\def\PY@tc##1{\textcolor[rgb]{0.50,0.00,0.50}{##1}}}
\expandafter\def\csname PY@tok@gd\endcsname{\def\PY@tc##1{\textcolor[rgb]{0.63,0.00,0.00}{##1}}}
\expandafter\def\csname PY@tok@gi\endcsname{\def\PY@tc##1{\textcolor[rgb]{0.00,0.63,0.00}{##1}}}
\expandafter\def\csname PY@tok@gr\endcsname{\def\PY@tc##1{\textcolor[rgb]{1.00,0.00,0.00}{##1}}}
\expandafter\def\csname PY@tok@ge\endcsname{\let\PY@it=\textit}
\expandafter\def\csname PY@tok@gs\endcsname{\let\PY@bf=\textbf}
\expandafter\def\csname PY@tok@gp\endcsname{\let\PY@bf=\textbf\def\PY@tc##1{\textcolor[rgb]{0.00,0.00,0.50}{##1}}}
\expandafter\def\csname PY@tok@go\endcsname{\def\PY@tc##1{\textcolor[rgb]{0.53,0.53,0.53}{##1}}}
\expandafter\def\csname PY@tok@gt\endcsname{\def\PY@tc##1{\textcolor[rgb]{0.00,0.27,0.87}{##1}}}
\expandafter\def\csname PY@tok@err\endcsname{\def\PY@bc##1{\setlength{\fboxsep}{0pt}\fcolorbox[rgb]{1.00,0.00,0.00}{1,1,1}{\strut ##1}}}
\expandafter\def\csname PY@tok@kc\endcsname{\let\PY@bf=\textbf\def\PY@tc##1{\textcolor[rgb]{0.00,0.50,0.00}{##1}}}
\expandafter\def\csname PY@tok@kd\endcsname{\let\PY@bf=\textbf\def\PY@tc##1{\textcolor[rgb]{0.00,0.50,0.00}{##1}}}
\expandafter\def\csname PY@tok@kn\endcsname{\let\PY@bf=\textbf\def\PY@tc##1{\textcolor[rgb]{0.00,0.50,0.00}{##1}}}
\expandafter\def\csname PY@tok@kr\endcsname{\let\PY@bf=\textbf\def\PY@tc##1{\textcolor[rgb]{0.00,0.50,0.00}{##1}}}
\expandafter\def\csname PY@tok@bp\endcsname{\def\PY@tc##1{\textcolor[rgb]{0.00,0.50,0.00}{##1}}}
\expandafter\def\csname PY@tok@fm\endcsname{\def\PY@tc##1{\textcolor[rgb]{0.00,0.00,1.00}{##1}}}
\expandafter\def\csname PY@tok@vc\endcsname{\def\PY@tc##1{\textcolor[rgb]{0.10,0.09,0.49}{##1}}}
\expandafter\def\csname PY@tok@vg\endcsname{\def\PY@tc##1{\textcolor[rgb]{0.10,0.09,0.49}{##1}}}
\expandafter\def\csname PY@tok@vi\endcsname{\def\PY@tc##1{\textcolor[rgb]{0.10,0.09,0.49}{##1}}}
\expandafter\def\csname PY@tok@vm\endcsname{\def\PY@tc##1{\textcolor[rgb]{0.10,0.09,0.49}{##1}}}
\expandafter\def\csname PY@tok@sa\endcsname{\def\PY@tc##1{\textcolor[rgb]{0.73,0.13,0.13}{##1}}}
\expandafter\def\csname PY@tok@sb\endcsname{\def\PY@tc##1{\textcolor[rgb]{0.73,0.13,0.13}{##1}}}
\expandafter\def\csname PY@tok@sc\endcsname{\def\PY@tc##1{\textcolor[rgb]{0.73,0.13,0.13}{##1}}}
\expandafter\def\csname PY@tok@dl\endcsname{\def\PY@tc##1{\textcolor[rgb]{0.73,0.13,0.13}{##1}}}
\expandafter\def\csname PY@tok@s2\endcsname{\def\PY@tc##1{\textcolor[rgb]{0.73,0.13,0.13}{##1}}}
\expandafter\def\csname PY@tok@sh\endcsname{\def\PY@tc##1{\textcolor[rgb]{0.73,0.13,0.13}{##1}}}
\expandafter\def\csname PY@tok@s1\endcsname{\def\PY@tc##1{\textcolor[rgb]{0.73,0.13,0.13}{##1}}}
\expandafter\def\csname PY@tok@mb\endcsname{\def\PY@tc##1{\textcolor[rgb]{0.40,0.40,0.40}{##1}}}
\expandafter\def\csname PY@tok@mf\endcsname{\def\PY@tc##1{\textcolor[rgb]{0.40,0.40,0.40}{##1}}}
\expandafter\def\csname PY@tok@mh\endcsname{\def\PY@tc##1{\textcolor[rgb]{0.40,0.40,0.40}{##1}}}
\expandafter\def\csname PY@tok@mi\endcsname{\def\PY@tc##1{\textcolor[rgb]{0.40,0.40,0.40}{##1}}}
\expandafter\def\csname PY@tok@il\endcsname{\def\PY@tc##1{\textcolor[rgb]{0.40,0.40,0.40}{##1}}}
\expandafter\def\csname PY@tok@mo\endcsname{\def\PY@tc##1{\textcolor[rgb]{0.40,0.40,0.40}{##1}}}
\expandafter\def\csname PY@tok@ch\endcsname{\let\PY@it=\textit\def\PY@tc##1{\textcolor[rgb]{0.25,0.50,0.50}{##1}}}
\expandafter\def\csname PY@tok@cm\endcsname{\let\PY@it=\textit\def\PY@tc##1{\textcolor[rgb]{0.25,0.50,0.50}{##1}}}
\expandafter\def\csname PY@tok@cpf\endcsname{\let\PY@it=\textit\def\PY@tc##1{\textcolor[rgb]{0.25,0.50,0.50}{##1}}}
\expandafter\def\csname PY@tok@c1\endcsname{\let\PY@it=\textit\def\PY@tc##1{\textcolor[rgb]{0.25,0.50,0.50}{##1}}}
\expandafter\def\csname PY@tok@cs\endcsname{\let\PY@it=\textit\def\PY@tc##1{\textcolor[rgb]{0.25,0.50,0.50}{##1}}}

\def\PYZbs{\char`\\}
\def\PYZus{\char`\_}
\def\PYZob{\char`\{}
\def\PYZcb{\char`\}}
\def\PYZca{\char`\^}
\def\PYZam{\char`\&}
\def\PYZlt{\char`\<}
\def\PYZgt{\char`\>}
\def\PYZsh{\char`\#}
\def\PYZpc{\char`\%}
\def\PYZdl{\char`\$}
\def\PYZhy{\char`\-}
\def\PYZsq{\char`\'}
\def\PYZdq{\char`\"}
\def\PYZti{\char`\~}
% for compatibility with earlier versions
\def\PYZat{@}
\def\PYZlb{[}
\def\PYZrb{]}
\makeatother


    % Exact colors from NB
    \definecolor{incolor}{rgb}{0.0, 0.0, 0.5}
    \definecolor{outcolor}{rgb}{0.545, 0.0, 0.0}



    
    % Prevent overflowing lines due to hard-to-break entities
    \sloppy 
    % Setup hyperref package
    \hypersetup{
      breaklinks=true,  % so long urls are correctly broken across lines
      colorlinks=true,
      urlcolor=urlcolor,
      linkcolor=linkcolor,
      citecolor=citecolor,
      }
    % Slightly bigger margins than the latex defaults
    
    \geometry{verbose,tmargin=1in,bmargin=1in,lmargin=1in,rmargin=1in}
    
    

    \begin{document}
    
    
    \maketitle
    
    

    
    Table of Contents{}

{{7~~}Moving Beyond Linearity}

{{7.1~~}Polynomial Regression}

{{7.2~~}Step Functions}

{{7.3~~}Basis Functions}

{{7.4~~}Regression Splines}

{{7.4.1~~}Piecewise Polynomials}

{{7.4.2~~}Constraints and Splines}

{{7.4.3~~}The Spline Basis Representation}

{{7.4.4~~}Choosing the Number and the Locations of the Knots}

{{7.4.5~~}Comparison to Polynomial Regression}

{{7.5~~}Smoothing Splines}

{{7.5.1~~}An Overview of Smoothing Splines}

{{7.5.2~~}Choosing the Smoothing Parameter \(\lambda\)}

{{7.6~~}Local Regression}

{{7.7~~}Generalized Additive Models}

{{7.7.1~~}GAMs for Regression Problems}

{{7.7.2~~}GAMs for Classification Problems}

{{7.8~~}Footnotes}

    \begin{center}\rule{0.5\linewidth}{\linethickness}\end{center}

\hypertarget{moving-beyond-linearity}{%
\section{Moving Beyond Linearity}\label{moving-beyond-linearity}}

\begin{center}\rule{0.5\linewidth}{\linethickness}\end{center}

    \hypertarget{polynomial-regression}{%
\subsection{Polynomial Regression}\label{polynomial-regression}}

    \begin{itemize}
\tightlist
\item
  \textbf{\emph{Simple polynomial regression}} is a regression model
  which is polynomial54 in the feature variable X
\end{itemize}

\[Y = \beta_0 + \sum_{i = 1}^d \beta_iX^d\] - The model can be fit as a
simple linear regression model with predictors
\(X_1, \dots, X_d = X, \dots X^d\). - It is rare to take
\(d \geqslant 4\) because it lead strange curves

    \hypertarget{advantages}{%
\subparagraph{Advantages}\label{advantages}}

    \begin{itemize}
\tightlist
\item
  Interpretability
\item
  More flexibility than linear regression, can better model non-linear
  relationships
\end{itemize}

    \hypertarget{disadvantages}{%
\subparagraph{Disadvantages}\label{disadvantages}}

    \begin{itemize}
\tightlist
\item
  Greater flexibility can lead to overfitting (can be mitigating by
  keeping \(d\) low)
\item
  Imposes global structure on target function (as does linear
  regression)
\end{itemize}

    \hypertarget{step-functions}{%
\subsection{Step Functions}\label{step-functions}}

    \begin{itemize}
\tightlist
\item
  Step functions model the target function as locally constant by
  converting the continuous variable \(X\) into an \textbf{\emph{ordered
  categorical variable}}.as follows

  \begin{itemize}
  \tightlist
  \item
    Choose \(K\) points \(c_1, \dots, c_K \in [\min(X), \max(X)]\)
  \item
    Define \(K + 1\) ``dummy'' variables \begin{align*}
      C_0(X) &= I(X < c_1)\\
      C_i(X) &= I(c_i \leqslant X < c_{i+1})\qquad 1 \leqslant i \leqslant K - 1\\
      C_K(X) &= I(c_K \leqslant X)
      \end{align*}
  \item
    Fit a linear regression model to the predictors
    \(C_1, \dots, C_K\)55
  \end{itemize}
\end{itemize}

    \hypertarget{advantages}{%
\subparagraph{Advantages}\label{advantages}}

    \begin{itemize}
\tightlist
\item
  Flexibility to model non-linear relationships
\item
  Can model local behavior better than global models (e.g.~linear and
  polynomial regression)
\end{itemize}

    \hypertarget{disadvantages}{%
\subparagraph{Disadvantages}\label{disadvantages}}

    \begin{itemize}
\tightlist
\item
  Locally constant assumption is strong, breakpoints in data may not be
  realized.
\end{itemize}

    \hypertarget{basis-functions}{%
\subsection{Basis Functions}\label{basis-functions}}

    In general, we can fit a regression model

\[Y = \beta_0 + \sum_{i=1}^Kb_i(X)\]

where the \(b_i(X)\) are called \textbf{\emph{basis functions}} 56

    \hypertarget{advantages}{%
\subparagraph{Advantages}\label{advantages}}

    Different choices of basis functions are useful for modeling different
types of relationships (for example, Fourier basis functions can model
periodic behavior).

    \hypertarget{disadvantages}{%
\subparagraph{Disadvantages}\label{disadvantages}}

    \begin{itemize}
\tightlist
\item
  As usual, greater flexibility can lead to overfitting
\item
  Some choices of basis functions (i.e.~basis functions which are not
  suited to the assumed true functional relationship) will likely have
  poor performance.
\end{itemize}

    \hypertarget{regression-splines}{%
\subsection{Regression Splines}\label{regression-splines}}

    \textbf{\emph{Regression splines}} are a flexible (and common choice of)
class of basis functions which extend both polynomial and piecewise
constant basis functions.

    \hypertarget{piecewise-polynomials}{%
\subsubsection{Piecewise Polynomials}\label{piecewise-polynomials}}

    \textbf{\emph{Piecewise polynomials}} fit separate low-degree
polynomials over different regions of \(X\). The points where the
coefficients change are called \textbf{\emph{knots}}.

    \hypertarget{advantages}{%
\subparagraph{Advantages}\label{advantages}}

    \begin{itemize}
\tightlist
\item
  Flexibility to model non-linear relationships (as with all non-linear
  methods discussed in this chapter)
\item
  Sensitivity to local behavior (less rigid than global model).
\end{itemize}

    \hypertarget{disadvantages}{%
\subparagraph{Disadvantages}\label{disadvantages}}

    \begin{itemize}
\tightlist
\item
  Overly flexible - each piece has independent degrees of freedom
\item
  Can have unnatural breaks at knots without appropriate constraints
\item
  Possibility of overfitting (as with all non-linear methods discussed
  in this chapter)
\end{itemize}

    \hypertarget{constraints-and-splines}{%
\subsubsection{Constraints and Splines}\label{constraints-and-splines}}

    \begin{itemize}
\tightlist
\item
  To remedy overflexibility of piecewise polynomials, we can impose
  constraints at the knots, e.g.~continuity, differentiability of
  various orders (smoothness).
\item
  A \textbf{\emph{spline}} is a piecewise degree \(d\) polynomial that
  has continuous derivatives up to order \(d-1\) at each knot (hence
  everywhere).
\end{itemize}

    \hypertarget{advantages}{%
\subparagraph{Advantages}\label{advantages}}

    \begin{itemize}
\tightlist
\item
  Same advantages to piecewise polynomials, while improving on the
  disadvantages
\end{itemize}

    \hypertarget{disadvantages}{%
\subparagraph{Disadvantages}\label{disadvantages}}

    \begin{itemize}
\tightlist
\item
  Overfitting
\item
  Poor match to the true relationship
\end{itemize}

    \hypertarget{the-spline-basis-representation}{%
\subsubsection{The Spline Basis
Representation}\label{the-spline-basis-representation}}

    \begin{itemize}
\tightlist
\item
  Regression splines can be modeled using an appropriate basis, of which
  there are many choices.
\item
  For example, we can model a \(d\) degree spline with \(K\) knots using
  \textbf{\emph{truncated power basis}}
  \[b_1(X), \dots, b_{K+d}(X) = x, \dots, x^d, h(X, \xi_1), \dots, h(X, \xi_K)\]
  where \(\xi_i\) is the \(i-th\) knot and \[h(X - \xi_i) =
    \begin{cases} 
        (X-\xi_i)^d & X > \xi_i\\
        0 & X \leqslant \xi_i
    \end{cases}\] is the \textbf{\emph{truncated power function}} of
  degree \(d\).
\end{itemize}

    \hypertarget{advantages}{%
\subparagraph{Advantages}\label{advantages}}

    Ibid.

    \hypertarget{disadvantages}{%
\subparagraph{Disadvantages}\label{disadvantages}}

    Beyond those mentioned above, splines can have a high variance near
\(\min(X), \max(X)\) (this can be overcome by using
\textbf{\emph{natural splines}} which impose boundary constraints, i.e
constraints on the form of the model on \([\min(X), \xi_1]\),
\([\max(X), \xi_K]\) (e.g.~linearity)

    \hypertarget{choosing-the-number-and-the-locations-of-the-knots}{%
\subsubsection{Choosing the Number and the Locations of the
Knots}\label{choosing-the-number-and-the-locations-of-the-knots}}

    \begin{itemize}
\tightlist
\item
  In practice, we place knots in uniform fashion, e.g.~by specifying the
  desired degrees of freedom and using software to place the knots at
  uniform quantiles of the data.
\item
  The desired degrees of freedom (hence number of knots) can be obtained
  using cross-validation.
\end{itemize}

    \hypertarget{comparison-to-polynomial-regression}{%
\subsubsection{Comparison to Polynomial
Regression}\label{comparison-to-polynomial-regression}}

    Often gives superior results to polynomial regression -- the latter must
use higher degrees (imposing global structure) while the former can
increase the number of knots while leaving the degree fixed (sensitivity
to local behavior) as well as varying the density of knots (i.e.~placing
more where the response varies rapidly, less where it is more stable)

    \hypertarget{smoothing-splines}{%
\subsection{Smoothing Splines}\label{smoothing-splines}}

    \hypertarget{an-overview-of-smoothing-splines}{%
\subsubsection{An Overview of Smoothing
Splines}\label{an-overview-of-smoothing-splines}}

    \begin{itemize}
\tightlist
\item
  A \textbf{\emph{smoothing spline}} 57 is a function
\end{itemize}

\[\hat{g}_\lambda = \underset{g}{\text{argmin}\,}\sum_{i=1}^n(y_i - g(x_i))^2 + \lambda \int g''(t)^2\,dt\]
where \(\lambda = 0\) is a tuning parameter58 - \(\lambda\) controls the
bias-variance tradeoff. \(\lambda = 0\) corresponds to the
\textbf{\emph{interpolation spline}} which fits all the data points
exactly and will be thus woefull overfit. In the limit
\(\lambda \rightarrow \infty\), \(\hat{g}_\lambda\) approaches the least
squares line - It can be show that the function \(\hat{g}_\lambda\) is a
piecewise cubic polynomial with knots at the unique \(x_i\) and
continuous first and second derivatives at the knots 59

    \hypertarget{choosing-the-smoothing-parameter-lambda}{%
\subsubsection{\texorpdfstring{Choosing the Smoothing Parameter
\(\lambda\)}{Choosing the Smoothing Parameter \textbackslash{}lambda}}\label{choosing-the-smoothing-parameter-lambda}}

    \begin{itemize}
\tightlist
\item
  The parameter \(\lambda\) controls the \textbf{\emph{effective degrees
  of freedom}} \(df_{\lambda}\). As \(\lambda\) goes from \$0 \$ to
  \(\infty\), \(df_\lambda\) goes from \(n\) to \(2\).
\item
  The effective degress of freedom is defined to be
  \[df_\lambda = \text{trace}(S_\lambda)\] where \(S_\lambda\) is the
  matrix such that \(\mathbf{\hat{g}}_\lambda = S_\lambda \mathbf{y}\)
  where \(\mathbf{\hat{g}}\) is the vector of fitted values.
\item
  \(\lambda\) can be chosen by cross-validation. LOOCV is particularly
  efficient to compute 60
\end{itemize}

\[RSS_{cv}(\lambda) = \sum_{i=1}^n (y_i - \hat{g}_\lambda^{(-i)}(x_i))^2 = \sum_{i=1}^n\left(\frac{y_i - \hat{g}_\lambda(x_i)}{1-tr(S_{\lambda})}\right)^2 \]

    \hypertarget{advantages}{%
\subparagraph{Advantages}\label{advantages}}

    \begin{itemize}
\tightlist
\item
  Flexibility/nonlinearity
\item
  As a shrinkage method, effective degrees of freedom are reduced,
  helping to balance bias-variance tradeoff and avoid overfitting.
\end{itemize}

    \hypertarget{disadvantages}{%
\subparagraph{Disadvantages}\label{disadvantages}}

    \begin{itemize}
\tightlist
\item
  As usual, flexibility can lead to overfitting
\end{itemize}

    \hypertarget{local-regression}{%
\subsection{Local Regression}\label{local-regression}}

    \begin{itemize}
\tightlist
\item
  Computes the fit at a target point by regressing on nearby training
  observations
\item
  Is \textbf{\emph{memory-based}} - all the training data is necessary
  for computing a prediction
\item
  In multiple linear regression, \textbf{\emph{variable coefficient
  models}} fit global regression to some variables and local to others
\end{itemize}

    \hypertarget{algorithm-k-nearest-neighbors-regression}{%
\subparagraph{\texorpdfstring{Algorithm: \(K\)-nearest neighbors
regression}{Algorithm: K-nearest neighbors regression}}\label{algorithm-k-nearest-neighbors-regression}}

    Fix the parameter61 \(1 \leqslant k \leqslant n\). For each \(X_=x_0\):
1. Get the neighborhood \(N_{i0}= \{k\ \text{closest}\ x_i\}\). 2.
Assign a weight \(K_{i0} = K(x_i, x_0)\) to each point \(x_i\) such that
such that - each point outside \(x_i\notin N_{i0}\) has
\(K_{i0}(x_i)=0\). - the furthest point \(x_i\in N_{i0}\) has weight
zero - the closest point \(x_i\in N_{i0}\) has the highest weight. 3.
Fit a weighted least squares regression

\[ (\hat{\beta_0}, \hat{\beta_1}) = \sum_{i=1}^nK_{i0}(y_i - \beta_0 - \beta_1 x_i)^2\]
4. Predict \(\hat{f}(x_0) = \hat{\beta_0} + \hat{\beta_1}x_0\).

    \hypertarget{generalized-additive-models}{%
\subsection{Generalized Additive
Models}\label{generalized-additive-models}}

    A \textbf{\emph{Generalized additive model}} is a model which is a sum
of nonlinear functions of the individual predictors.

    \hypertarget{gams-for-regression-problems}{%
\subsubsection{GAMs for Regression
Problems}\label{gams-for-regression-problems}}

    \begin{itemize}
\tightlist
\item
  A GAM for regression 62 is a model
\end{itemize}

\[Y =\beta_0 + \sum_{j=1}^p f_j(X_j) + \epsilon\]

where the functions \(f_j\) are smooth non-linear functions.

\begin{itemize}
\tightlist
\item
  GAMs can be used to combine methods from this chapter -- one can fit
  different nonlinear functions \(f_j\) to the predictors \(X_j\) 63
\item
  Standard software can fit GAMs with smoothing splines via
  \href{https://en.wikipedia.org/wiki/Backfitting_algorithm}{\textbf{\emph{backfitting}}}
\end{itemize}

    \hypertarget{advantages}{%
\subparagraph{Advantages}\label{advantages}}

    \begin{itemize}
\tightlist
\item
  Nonlinearity hence flexibility
\item
  Automatically introduces nonlinearity - obviates the need to
  experiment with different nonlinear transformations
\item
  Interpretability/inference - the \(f_j\) allow to consider the effect
  of each feature \(X_j\) independently of the others.
\item
  Smoothness of individual \(f_j\) can be summarized via degrees of
  freedom.
\item
  Represents a nice compromise betwee linear and fully non-parametric
  models (see \href{}{§8}).
\end{itemize}

    \hypertarget{disadvantages}{%
\subparagraph{Disadvantages}\label{disadvantages}}

    \begin{itemize}
\tightlist
\item
  Usual disadvantages of nonlinearity
\item
  Doesn't allow for interactions between features (this can be overcome
  by including nonlinear functios of the interaction terms
  \(f(X_j,X_k)\)
\item
  The additive constraint is strong, restricts flexibility.
\end{itemize}

    \hypertarget{gams-for-classification-problems}{%
\subsubsection{GAMs for Classification
Problems}\label{gams-for-classification-problems}}

    GAMs can be used for classification. For example, a GAM for logistic
regression is

\[\log\left(\frac{p_k(X)}{1 - p_k(X)}\right) =\beta_0 + \sum_{j=1}^p f_j(X_j) + \epsilon\]

where \(p_k(X) =\text{Pr}(Y = k\ |\ X)\).

    \begin{center}\rule{0.5\linewidth}{\linethickness}\end{center}

\hypertarget{footnotes}{%
\subsection{Footnotes}\label{footnotes}}

    \hypertarget{foot54}{}
\begin{enumerate}
\def\labelenumi{\arabic{enumi}.}
\setcounter{enumi}{53}
\tightlist
\item
  In statistical literature, polynomial regression is sometimes referred
  to as linear regression. This is because the model is linear in the
  population parameters \(\beta_i\). ↩
\end{enumerate}

\hypertarget{foot55}{}
\begin{enumerate}
\def\labelenumi{\arabic{enumi}.}
\setcounter{enumi}{54}
\tightlist
\item
  The variable \(C_0(X)\) accounts for an intercept. Alternatively fit a
  linear model to \(C_0, \dots, C_K\) with no intercept. ↩
\end{enumerate}

\hypertarget{foot56}{}
\begin{enumerate}
\def\labelenumi{\arabic{enumi}.}
\setcounter{enumi}{55}
\tightlist
\item
  Such a model amounts to the assumption that the target function lives
  in a finite-dimensional subspace of the vector space of all functions
  \(f:X\rightarrow Y\). ↩
\end{enumerate}

\hypertarget{foot57}{}
\begin{enumerate}
\def\labelenumi{\arabic{enumi}.}
\setcounter{enumi}{56}
\tightlist
\item
  The function \(g\) is not guaranteed to be smooth in the sense of
  infinitely differentiable. The penalty on the second derivative
  (curvature) penalizes the ``roughness'' or ``wiggliness'' of \(g\),
  hence ``smoothes out'' noise in the data. Other penalties have been
  used ↩
\end{enumerate}

\hypertarget{foot58}{}
\begin{enumerate}
\def\labelenumi{\arabic{enumi}.}
\setcounter{enumi}{57}
\tightlist
\item
  A tuning parameter is also called a hyperparameter ↩
\end{enumerate}

\hypertarget{foot59}{}
\begin{enumerate}
\def\labelenumi{\arabic{enumi}.}
\setcounter{enumi}{58}
\tightlist
\item
  Thus \(\hat{g}\) is a natural cubic spline with knots at the \(x_i\).
  However, it is not the spline one obtains in
  Section \ref{the-spline-basis-representation}. It is a ``shrunken''
  version, where \(\lambda\) controls the shrinkage. ↩
\end{enumerate}

\hypertarget{foot60}{}
\begin{enumerate}
\def\labelenumi{\arabic{enumi}.}
\setcounter{enumi}{59}
\tightlist
\item
  Compare to a similar formula in §5.1.2 ↩
\end{enumerate}

\hypertarget{foot61}{}
\begin{enumerate}
\def\labelenumi{\arabic{enumi}.}
\setcounter{enumi}{60}
\tightlist
\item
  Our description of the algorithm deviates a bit from the book, but
  it's equivalent. ↩
\end{enumerate}

\hypertarget{foot62}{}
\begin{enumerate}
\def\labelenumi{\arabic{enumi}.}
\setcounter{enumi}{61}
\tightlist
\item
  ``Additive'' because we are summing the \(f_i\). ``Generalized''
  because it generalizes from the linear functions \(\beta_jX_j\) in
  ordinary linear regression. ↩
\end{enumerate}

\hypertarget{foot63}{}
\begin{enumerate}
\def\labelenumi{\arabic{enumi}.}
\setcounter{enumi}{62}
\tightlist
\item
  It's not hard to see that (with the exception of local regression),
  all the models discussed in this chapter can be seen as special cases
  of GAM.
\end{enumerate}

↩


    % Add a bibliography block to the postdoc
    
    
    
    \end{document}
